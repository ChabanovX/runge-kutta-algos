\documentclass{article}
\usepackage{amsmath}
\usepackage{amssymb}

\begin{document}

\section*{Practical Tasks}

\subsection*{Task 1}

Write a computer program that solves:

\[
\begin{aligned}
    & a) \quad y' = x + \cos y, \quad y(1) = 30, \quad 1 \leq x \leq 2, \\
    & b) \quad y' = x^2 + y^2, \quad y(2) = 1, \quad 1 \leq x \leq 2.
\end{aligned}
\]

Implement in the code numerical Runge-Kutta explicit methods of 2nd and 4th order with grid steps:
\( h = 0.1, 0.05, 0.01, 0.005, 0.001 \).

For each numerical method, compose the four arrays of the relative errors for the numbers \( N = 20, 100, 200, 1000 \) in the same \( x \) coordinates:

\[
\begin{aligned}
    & N = 20: \quad |x_k(n_{20}) - x_k(n_{10})|, \quad k = 1, 10, \\
    & N = 100: \quad |x_k(n_{100}) - x_k(n_{20})|, \quad k = 1, 10, \\
    & N = 200: \quad |x_k(n_{200}) - x_k(n_{100})|, \quad k = 1, 10, \\
    & N = 1000: \quad |x_k(n_{1000}) - x_k(n_{200})|, \quad k = 1, 10.
\end{aligned}
\]

Provide the functionality of printing the arrays of the relative errors onto the console in the format:

\begin{itemize}
    \item In the first line, print the title of the array (e.g., ``N\_20'').
    \item In the second line, print the values stored in the array, separated by spaces, without the space after the last item.
\end{itemize}

For each numerical method, compose the array of the absolute errors:

\[
\text{Absolute Errors} =
\max_{k=1,10}
\begin{cases}
    |x_k(n_{20}) - x_k(n_{10})|, & N = 20 \\
    |x_k(n_{100}) - x_k(n_{20})|, & N = 100 \\
    |x_k(n_{200}) - x_k(n_{100})|, & N = 200 \\
    |x_k(n_{1000}) - x_k(n_{200})|, & N = 1000.
\end{cases}
\]

Provide the functionality of plotting \(\log_2\) for the absolute errors of both methods together in one picture.

\subsection*{Task 2}

Write a computer program that solves:

\[
y'' = y \sin x, \quad y(0) = 0, \quad y'(0) = 1, \quad 0 \leq x \leq 1.
\]

Implement in the code numerical Runge-Kutta explicit methods of 2nd and 4th order with grid steps:
\( h = 0.1, 0.05, 0.01, 0.005, 0.001 \).

For each numerical method, compose the four arrays of the relative errors for the numbers \( N = 20, 100, 200, 1000 \) in the same \( x \) coordinates:

\[
\begin{aligned}
    & N = 20: \quad |x_k(n_{20}) - x_k(n_{10})|, \quad k = 1, 10, \\
    & N = 100: \quad |x_k(n_{100}) - x_k(n_{20})|, \quad k = 1, 10, \\
    & N = 200: \quad |x_k(n_{200}) - x_k(n_{100})|, \quad k = 1, 10, \\
    & N = 1000: \quad |x_k(n_{1000}) - x_k(n_{200})|, \quad k = 1, 10.
\end{aligned}
\]

Provide the functionality of printing the arrays of the relative errors onto the console in the format:

\begin{itemize}
    \item In the first line, print the title of the array (e.g., ``N\_20'').
    \item In the second line, print the values stored in the array, separated by spaces, without the space after the last item.
\end{itemize}

For each numerical method, compose the array of the absolute errors:

\[
\text{Absolute Errors} =
\max_{k=1,10}
\begin{cases}
    |x_k(n_{20}) - x_k(n_{10})|, & N = 20 \\
    |x_k(n_{100}) - x_k(n_{20})|, & N = 100 \\
    |x_k(n_{200}) - x_k(n_{100})|, & N = 200 \\
    |x_k(n_{1000}) - x_k(n_{200})|, & N = 1000.
\end{cases}
\]

Provide the functionality of plotting \(\log_2\) for the absolute errors of both methods together in one picture.

\end{document}
``` 

Let me know if you need any further customization or additions!